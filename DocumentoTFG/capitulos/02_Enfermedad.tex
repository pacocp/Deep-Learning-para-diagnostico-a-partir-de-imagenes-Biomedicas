\chapter{Enfermedad de Alzheimer}

\section{Alzheimer}

La \textbf{enfermedad de Alzheimer (EA)}  es una enfermedad neurodegenerativa que se manifiesta como deterioro cognitivo y trastornos conductuales. Se caracteriza, en la mayoría de ocasiones,  por una perdida de memoria inmediata  y de otras capacidades mentales, a medida que mueren las células nerviosas(neuronas) y se atrofian diferentes zonas del cerebro. La enfermedad suele tener una duración media de aproximada después del diagnóstico al paciente de unos 10 años, todo dependiendo por supuesto de la etapa en la que se diagnostique la enfermedad.
\\
La enfermedad de Alzheimer es la forma más común de demencia, es incurable y terminal, y aparece con mayor frecuencia en personas mayores de 65 años de edad \cite{3}, aunque en otros casos más raros y extremos puede ser desarrollada a partir de los 40 años. 
\\
Los síntomas de la enfermedad como una enfermedad nosológica fueron definidos por primera vez por \textbf{Emil Kraepelin} (Neustrelitz, 15 de febrero de 1856 – Múnich, 7 de octubre de 1926), mientras que la neuropatología característica fue observada por primera vez por \textbf{Alois Alzheimer} (Marktbreit, 14 de junio de 1864 - Breslavia, 19 de diciembre de 1915) en 1906. Ambos trabajaban en el mismo laboratorio, por lo que se puede considerar que el descubrimiento fue obra de ambos psiquiatras. Pero dado a que Kraepelin daba mucha importancia a encontrar la base neuropatológica de los desordenes psiquiátricos, decidió nombrar la enfermedad Alzheimer en honor a su compañero.
\\

Normalmente, el síntoma inicial es una perdida de la habilidad para adquirir nuevos recuerdos, y esto suele confundirse con actitudes relacionadas con la vejez o el estrés, ya que es en esta etapa de la vida en la que se suele desarrollar la enfermedad.Ante la sospecha de alzhéimer, el diagnóstico se realiza con evaluaciones de conductas cognitivas, así como neuroimágenes, de estar estas disponibles.
\\

A medida que la enfermedad va progresando, aparecen la confusión mental,irratibilidad y agresión, cambios de humor, transtornos del lenguaje, pérdida de la memoria a corto plazo y una predisposición a aislarse a medida que declinan los sentidos del paciente. Finalmente se pierden funciones biológicas, lo que conlleva la muerte.
\\

La causa de la enfermedad permanece desconocida \label{desconocida}, aunque las últimas investigaciones parece indicar que se encuentran implicados procesos de tipo priórico\cite{4}.  Las investigaciones suelen asociar la enfermedad a la aparición de placas seniles y ovillos neurofibrilares. Hoy en día, los tratamientos que se ofrecen moderados beneficios sintomáticos, pero no hay tratamiento que retrase o detenga el progreso de la enfermedad.\\

Según el siguiente estudio realizado en España \cite{populationofalzheimerinSpain}, en España podemos encontrar la siguiente tabla con respecto a la población que padece esta enfermedad:
\begin{table}[h]
	\centering
	\caption{Incidencias de la enfermedad en España respecto a la edad \cite{populationofalzheimerinSpain}}
	\label{tabla1}
	\begin{tabular}{|l|l|}
		\hline
		Edad  & \begin{tabular}[c]{@{}l@{}}Indidencia\\ (nuevos casos)\\ por cada mil personas\end{tabular} \\ \hline
		65-69 & 3                                                                                           \\ \hline
		70-74 & 6                                                                                           \\ \hline
		75-79 & 9                                                                                           \\ \hline
		80-84 & 23                                                                                          \\ \hline
		85-89 & 40                                                                                          \\ \hline
		90-   & 69                                                                                          \\ \hline
	\end{tabular}
\end{table}



\bibliography{bibliografia/bibliografia.bib}
\bibliographystyle{ieeetr} % hay varias formas de citar