\chapter{Introducción}

El objetivo de este proyecto es la predicción temprana de la enfermedad de Alzheimer a partir de imágenes biomédicas de pacientes en distintos estados de la endermedad. Este proceso de predicción se realizará utilizando técnicas que se engloban dentro del Deep Learning.

\section{Motivación}

Actualmente, la enfermadad de Alzheimer es la principal causa de demencia en el ser humano. Lo que se produce es un deterioro progresivo de las celulas nerviosas cerebrales, lo que desemboca en demencia senil. El aumento de Apersonas afectadas por esta enfermedad viene dado en parte por el aumento de la esperanza de vida, ya que esta enfermedad se da en personas de edad avanzada. Las consecuencias de esta enfermedad son varias, pero todas limitan la calidad de vida de la persona que la padece, además del de las personas de su entorno, ya que esta enfermedad va minando la autonomía del enfermo sobre su propio cuerpo. Esto provocará que el enfermo necesite un apoyo externo para poder realizar su día a día.\\
Aún no se sabe la causa de la enfermedad de Alzheimer, aunque si se conocen factores de riesgo que pueden ayudar a su desarrollo \cite{RiskFactors}. Los principales que se conocen son los siguientes: la edad (como se ha explicado antes con el aumento de la esperanza de vida), el sexo (afectando más a mujeres que a hombres), el nivel de eduación (las personas con estudios y mayor actividad intelectual son menos propensas a esta enfermedad), los trastornos metabólicos asociados con la resistencia a la insulina, la obesidad, hipertensión o diabetes y la genética (la presencia del genotipo determinado del gen de la EPOE) entre otros factores.\\
No existe una cura de la enfermedad, solo se conocen tratamientos que relentizan su avance. Estos tratamientos deben realizarse en las primeras fases de la enfermedad, ya que después pueden resultar perjudiciales para el paciente.\\
El diagnóstico del Alzheimer se podría dividir en tres partes \cite{Diagnostico}:
\begin{itemize}
	\item \textbf{Evaluación de estado de ánimo y estado mental:} El estado mental se evalúa para darle al médico una idea general de qué tan bien está funcionando la mente. Este examen da un sentido general sobre si la persona: está consciente de que tiene síntomas. sabe la fecha, la hora, y dónde está ella, puede recordar una lista corta de palabras, seguir instrucciones y hacer cálculos simples. El doctor puede que le pregunte a la persona su dirección, qué año es y quién es el presidente del país. Puede que al individuo se le pida que deletree una palabra al revés, que dibuje un reloj y que copie un diseño. El doctor también puede evaluar el estado de ánimo y el sentimiento de bienestar de la persona para detectar si hay depresión u otra enfermedad que puede causar pérdida de memoria y confusión.
	\item \textbf{Examen físico y exámenes para el diagnóstico}: El doctor hará ciertos procedimientos para evaluar la salud en general de la persona como su dieta, tomar la presión arterial, o escuchar su corazón. Se harán pruebas de sangre y de orina y posiblemente se ordenen otros exámenes de laboratorio. La información que pueden proporcionar estos exámenes puede ayudar a identificar problemas como anemia, diabetes, problemas de los riñones o del hígado, deficiencias de vitaminas, anormalidades de la tiroides y problemas del corazón o de los vasos sanguíneos. Todas estas condiciones pueden causar confusión, problemas de memoria u otros síntomas similares a la Demencia.
	\item \textbf{Examen neurológico:} Un doctor o a veces un Neurólogo que se especializa en problemas del cerebro y del sistema nervioso, evaluará muy cuidadosamente a la persona para determinar si hay señales de otro tipo de problema del cerebro que no es Alzheimer. El doctor también va a evaluar los reflejos de la persona, el equilibrio, movimiento de los ojos, lenguaje y sensibilidad. El doctor está buscando señales de derrames pequeños o grandes, enfermedad de Parkinson, tumores cerebrales, acumulación de líquido en el cerebro y otros males que pueden perjudicar la memoria o la capacidad de pensar. El examen neurológico puede incluir un MRI (Imagen por Resonancia Magnética) o CT (tomografía computarizada). Los MRI y CT pueden revelar tumores, evidencia de derrames pequeños o grandes, daño a causa de lesiones por traumas severos de cabeza o acumulación de líquido. Medicare cubre el PET (tomografía por emisión de positrones) como una ayuda para el diagnóstico en ciertos casos .
\end{itemize}
El análisis de la imagen de resonancia magnética a través del ojo humano resulta sencillo si las alteraciones estructurales son apreciables. Normalmente se utiliza una escala de grises para poder resaltar las diferencias \cite{FormatosImagenes}. El número de bits con el que trabajan estos sistemas suele ser de 16 bits \cite{FormatosImagenes} lo cuál da una escala de grises de hasta 65536 niveles, muy superior a los niveles que el ojo puede diferenciar.\\
Para tener un estudio más exhaustivo del estado del paciente, se aplican una serie de algoritmos de ordenador para poder extraer las zonas de mayor interés que son las que observa el evaluador para poder determinar si existen indicios de la enfermedad en el paciente. El principal problema es que si nos encontramos en el inicio de la enfermedad, pueden no existir una alteración suficiente para poder diagnosticar la enfermedad por el ojo humano. He aquí la motivación de la realización del proyecto, poder ayudar a un diagnóstico temprano de la enfermadad para poder tratarla con procedimientos que reduzcan su avance, y mejoren la vida del paciente.

\section{Objetivos}

El objetivo que se persigue con este proyecto es la predicción temprana de la enfermedad del Alzheimer en personas que la comienzan a padecer. El estudio que se realiza puede dividirse en los siguientes objetivos específicos:
\begin{itemize}
	\item \textbf{Conversión de las imágenes para su tratamiento:} Las imágenes de la base de datos se encuentran en formato NiFTI. Para su tratamiento con los algoritmos, debemos convertirlas a un formato con el que se puedan tratar. Con esto se creará la base de datos de la cual se aprenderá para realizar las predicciones.
	\item \textbf{Elección de la arquitectura de la Convolutional Neural Network:} se deberá elegir una arquitectura para la red neuronal.
	\item \textbf{Entrenamiento de la red neuronal:} Entrenamiento de la red neuronal con la base de datos que se ha obtenido previamente.
	\item \textbf{Análisis de los resultados de clasificación y predicción:} En este punto, se analizarán los resultados obtenidos, donde se podrá observar si las técnicas aplicadas tienen un buen rendimiento.
\end{itemize}
\section{Resumen y estructura del proyecto}
\bibliography{bibliografia/bibliografia.bib}
\bibliographystyle{ieeetr} % hay varias formas de citar