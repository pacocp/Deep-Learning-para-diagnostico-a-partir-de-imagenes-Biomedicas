\mychapter{2}{Objetivos}

El objetivo que se persigue con este proyecto es la clasificación de las imágenes de pacientes entre Alzheimer y Normal utilizando imágenes en 2 dimensiones además de determinar cuál es la capa del cerebro que mejores resultados presenta a la hora de realizar esta clasificación. \\

El estudio que se realiza puede dividirse en los siguientes objetivos específicos:
\begin{itemize}
	\item \textbf{Conversión de las imágenes para su tratamiento:} Las imágenes de la base de datos se encuentran en formato NiFTI. Para su tratamiento con los algoritmos, debemos convertirlas a un formato con el que estos puedan tratar. Con esto se creará la base de datos para su uso en los distintos experimentos.
	\item \textbf{Elección de la arquitectura de la Convolutional Neural Network:} se deberá elegir una arquitectura para la red neuronal basándonos en la literatura actual. Esta arquitectura debe tratar con imágenes en dos dimensiones, debido a la potencia de computación que exigen las redes tridimensionales.
	\item \textbf{Entrenamiento de la red neurona con diversos métodos de validación:} Entrenamiento de la red neuronal con la base de datos que se ha obtenido previamente. Para ello se utilizarán diversos métodos de validación, de forma que los resultados obtenidos puedan considerarse significativos. Además, se deberán realizar una serie de experimentos para determinar qué capas del cerebro son las más significativas a la hora de clasificar entre normal y alzheimer.
	\item \textbf{Análisis de los resultados de clasificación y predicción:} En este punto, se analizarán los resultados obtenidos, donde se podrá observar si las técnicas aplicadas tienen un buen rendimiento con respecto a la literatura que se puede encontrar, y cuál es la capa del cerebro que mejor rendimiento demuestra.
\end{itemize}
