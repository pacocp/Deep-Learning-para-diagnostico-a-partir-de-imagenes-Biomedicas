\chapter*{}
%\thispagestyle{empty}
%\cleardoublepage

%\thispagestyle{empty}

\begin{titlepage}
 
 
\setlength{\centeroffset}{-0.5\oddsidemargin}
\addtolength{\centeroffset}{0.5\evensidemargin}
\thispagestyle{empty}

\noindent\hspace*{\centeroffset}\begin{minipage}{\textwidth}

\centering
%\includegraphics[width=0.9\textwidth]{imagenes/logo_ugr.jpg}\\[1.4cm]

%\textsc{ \Large PROYECTO FIN DE CARRERA\\[0.2cm]}
%\textsc{ INGENIERÍA EN INFORMÁTICA}\\[1cm]
% Upper part of the page
% 

 \vspace{3.3cm}

%si el proyecto tiene logo poner aquí
\includegraphics{imagenes/logo.png} 
 \vspace{0.5cm}

% Title

{\Huge\bfseries Título del proyecto\\
}
\noindent\rule[-1ex]{\textwidth}{3pt}\\[3.5ex]
{\large\bfseries Subtítulo del proyecto.\\[4cm]}
\end{minipage}

\vspace{2.5cm}
\noindent\hspace*{\centeroffset}\begin{minipage}{\textwidth}
\centering

\textbf{Autor}\\ {Nombre Apellido1 Apellido2 (alumno)}\\[2.5ex]
\textbf{Directores}\\
{Nombre Apellido1 Apellido2 (tutor1)\\
Nombre Apellido1 Apellido2 (tutor2)}\\[2cm]
%\includegraphics[width=0.15\textwidth]{imagenes/tstc.png}\\[0.1cm]
%\textsc{Departamento de Teoría de la Señal, Telemática y Comunicaciones}\\
%\textsc{---}\\
%Granada, mes de 201
\end{minipage}
%\addtolength{\textwidth}{\centeroffset}
\vspace{\stretch{2}}

 
\end{titlepage}






\cleardoublepage
\thispagestyle{empty}

\begin{center}
{\large\bfseries Deep Learning para diagnóstico a partir de imágenes Biomédicas}\\
\end{center}
\begin{center}
Francisco Carrillo Pérez (alumno)\\
\end{center}

%\vspace{0.7cm}
\noindent{\textbf{Palabras clave}:DeepLearning, CNN, Alzheimer}\\

\vspace{0.7cm}
\noindent{\textbf{Resumen}}\\

La enfermedad de Alzheimer es una de las enfermedades que más afecta a pacientes de edad avanzada en todo el mundo. Su cura se desconoce, por lo que un diagnóstico precoz puede ayudar a mejorar notablemente la vida del paciente. El problema es que la clasificación de la enfermedad en edad temprana es una tarea complicada, además de que se puede confundir con otros deterioros que acaecen propios de la edad. Con los nuevos avances que se han obtenido en el uso de técnicas de Deep Learning en el área de Visión por Computador y clasificación de imágenes, el uso de estas técnicas para la clasificación de pacientes puede suponer una ayuda notable a la hora de que se pueda diagnosticar correctamente si un paciente está comenzando a desarrollar esta enfermedad, con lo que se podría tratar con suficiente tiempo. Con este trabajo, se intenta responder si con estas técnicas podemos realizar una clasificación correcta de imágenes 2D cerebrales de pacientes y de ser así cuáles serían las capas del cerebro más favorables a la hora de realizar esta clasificación.
\cleardoublepage


\thispagestyle{empty}


\begin{center}
{\large\bfseries Deep Learning for diagnosis using Biomedical images}\\
\end{center}
\begin{center}
Francisco Carrillo Pérez (student)\\
\end{center}

%\vspace{0.7cm}
\noindent{\textbf{Keywords}: DeepLearning, CNN, Alzheimer}\\

\vspace{0.7cm}
\noindent{\textbf{Abstract}}\\

Alzheimer disease is one of the diseases that affect the most to the elderly in the whole world. The cure is unknown, so an early diagnosis is very important for making improves in the pacient life. The problem is that the diagnosis of the disease in an early stage is really difficult, adding the fact that it could be difficult to distinguish in respect to other damages of age. With the new development of Deep Learning techniques in the area of Computer Vision and Image Classification, the use of this techniques for pacient classification could be a usefull help for when deciding if a pacient is developing the disease, helping to treat it in an early stage. With this TFG, we are trying to anwer if witj this techniques we can make a good classification of 2D brain images of pacientes and if so, which are the best slices of the brain for doing this classification.
\chapter*{}
\thispagestyle{empty}

\noindent\rule[-1ex]{\textwidth}{2pt}\\[4.5ex]

Yo, \textbf{Francisco Carrillo Pérez}, alumno de la titulación Ingeniería Informática de la \textbf{Escuela Técnica Superior
de Ingenierías Informática y de Telecomunicación de la Universidad de Granada}, con 77140580Y, autorizo la
ubicación de la siguiente copia de mi Trabajo Fin de Grado en la biblioteca del centro para que pueda ser
consultada por las personas que lo deseen.

\vspace{6cm}

\noindent Fdo: Francisco Carrillo Pérez

\vspace{2cm}

\begin{flushright}
Granada a X de mes de 201 .
\end{flushright}


\chapter*{}
\thispagestyle{empty}

\noindent\rule[-1ex]{\textwidth}{2pt}\\[4.5ex]

D. \textbf{Luis Javier Herrera Maldonado (tutor1)}, Profesor del Área de XXXX del Departamento YYYY de la Universidad de Granada.

\vspace{0.5cm}

D. \textbf{ALberto Guillén perales (tutor2)}, Profesor del Área de XXXX del Departamento YYYY de la Universidad de Granada.


\vspace{0.5cm}

\textbf{Informan:}

\vspace{0.5cm}

Que el presente trabajo, titulado \textit{\textbf{Deep Learning para diagnóstico a partir de imágenes Biomédicas}},
ha sido realizado bajo su supervisión por \textbf{Francisco Carrillo Pérez (alumno)}, y autorizamos la defensa de dicho trabajo ante el tribunal
que corresponda.

\vspace{0.5cm}

Y para que conste, expiden y firman el presente informe en Granada a X de mes de 201 .

\vspace{1cm}

\textbf{Los directores:}

\vspace{5cm}

\noindent \textbf{Nombre Apellido1 Apellido2 (tutor1) \ \ \ \ \ Nombre Apellido1 Apellido2 (tutor2)}

\chapter*{Agradecimientos}
\thispagestyle{empty}

       \vspace{1cm}


Poner aquí agradecimientos...

